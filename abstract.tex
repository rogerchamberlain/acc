\begin{abstract}
The pairing of traditional multicore processors with accelerators
of various forms (e.g., graphics engines, reconfigurable logic)
can be referred to generally as \emph{architecturally diverse} systems.
Our interest in this work is \emph{truly} diverse systems, in which more than
one accelerator is used in the execution of an application.
These systems have the potential for substantial performance gains
relative to multicores alone; however, they pose significant difficulties
when it comes to application development. 

In spite of these difficulties, the use of accelerators in high performance
computation has grown substantially
over the past decade. This is primarily due to a pair of forces.  First,
with the demise of Dennard scaling, power has become a substantial limiting
factor in systems development, pushing computations to be more power
efficient (a strength of many accelerators). Second, the application development
environments for accelerators have improved substantially in recent years.

We review the use of multiple, distinct accelerators deployed in a individual
system or, more to the point, used concurrently within an individual
application. We give a history of architecturally diverse systems that
use multiple accelerators, discuss the motivations for diversity in
accelerators, and describe the approaches that both system designers and
application developers have used to put accelerators to beneficial use.
\end{abstract}

\begin{keyword}
graphics engine \sep GPU \sep reconfigurable logic \sep FPGA
\end{keyword}
