\section{Introduction}
\label{sec:intro}

The exploitation of Moore's Law~\cite{Moore65,Mack11}
and Dennard Scaling~\cite{Dennard74,Bohr07}
has enabled an unprecedented increase in computational capability over 
approximately five decades. Smaller feature sizes in semiconductor
technology, combined with greater power efficiency associated with each
transistor, have resulted in a long stretch of time in which application
performance greatly improved simply by waiting for the next generation
of processors to be shipped.

No longer. Since the early 2000s, the clock rates of individual processors
have stayed relatively constant, and improved performance has primarily
been due to a combination of speculative computation and parallelism.
Speculative computation has been used to increase the performance of
individual processor cores, and the number of cores on a processor chip
has been expanding for over a decade.

In addition to the transition of traditional processors into multicores,
the past two decades have also seen the advent of accelerators, or
\emph{architecturally diverse} systems.
Here, a multicore processor is augmented
with one or more specialized computational resources.  While most
commonly graphics engines or reconfigurable logic, other accelerators
that have seen use include digital signal processors
and vector processors (e.g., Intel's Xeon Phi).

Accelerators typically gain performance by exchanging generality for
specificity. While general purpose processor cores try to be very good
at everything, accelerators forego good performance on some computational
tasks so that they can excel on a more limited set of target tasks.
They accomplish this goal by specializing the computational data path,
each in their own specific way.

Graphics engines (or GPUs, graphics processing units) are now ubiquitous
in the Top500 supercomputer lists.\footnote{https://www.top500.org}
Graphics engines trade off single thread performance (i.e., minimum latency
to execute a sequential set of instructions) for greater computational
throughput on multiple threads.  A single fetch-decode unit controls a
collection of execution units, each of which performs the same instruction
on different data elements.  This enables the graphics engine to excel
at regular dense matrix computations (e.g., vector-matrix multiplication)
and other compute problems for which the same operation is to be performed
on a large set of data.

In 2015, Mittal and Vetter~\cite{mv15} surveyed techniques for effective
computation using a combination of multicores and graphics engines. They
consider systems in which the graphics engines are on a peripheral bus
(typically PCIe in modern systems) as well as fused systems (both compute
resources on a common chip).

Reconfigurable logic (or FPGAs, field-programmable logic arrays) has been
around for a longer time than graphics engines, but it has only recently come
to be seen as a viable option for high performance computation.
The unique advantage of reconfigurable logic is that the actual 
structure of the data path can be altered to match the needs of the specific
computation at hand.

Compton and Hauck~\cite{ch02} described the state of reconfigurable logic
targeting high performance computations back when the largest chips were
capable of about one million equivalent logic gates.  Since then, capacities
have increased 100-fold.  In addition, it is now common for reconfigurable
logic chips to also have some chip area dedicated to specific functions (e.g.,
memory, multipliers, etc.).

More recently, Trimberger~\cite{Trimberger15} provides a retrospective on
the first 30 years of reconfigurable logic technology, breaking it down into
three ``ages,'' which he calls \emph{invention}, \emph{expansion}, and
\emph{accumulation}.  Escobar et al.~\cite{ecv16} provide a comprehensive
analysis of the use of reconfigurable logic in high performance computing.
They extract features from families of high performance computing
applications and assess their suitability for hardware implementation.
Microsoft is now installing reconfigurable logic in its Azure cloud
systems~\cite{ccp+16}.

In this paper, we are interested in exploring \emph{truly architecturally
diverse} systems, by which we mean systems that include not just one, but
two (or potentially more) distinct accelerators.  Most of these systems
contain both graphics engines and reconfigurable logic, in addition to the
traditional multicore that forms the base of the system.

Here, we review the use of multiple accelerators used to execute an individual
application. We describe early efforts in this area and then concentrate on
how system designers and application developers have utilized multiple
accelerators.  Moving forward, what properties (of both the application and
the execution platform) are important in matching applications to platforms,
and how can we perform the appropriate deployments with little or no
explicit developer guidance.

The paper is organized as follows.  Section~\ref{sec:history} describes
the early history of multiple accelerator use (roughly defined as 2010
or before).  Section~\ref{sec:machines}
follows with a description of a number of machines that support multiple
accelerators, both early machines and more recent machines.
Section~\ref{sec:apps} lists a number of applications that have used
multiple accelerators in their execution.
Section~\ref{sec:dev} then describes development
environments that target multiple accelerator use, and
Section~\ref{sec:conclude} concludes and describes future work.
