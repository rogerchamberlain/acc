\section{Early History}
\label{sec:history}

Hardware acceleration of applications has a long history.
In 1984, Blank~\cite{Blank84}
published a survey of hardware accelerators used in computer-aided design 
that described a set of special-purpose engines going back nearly 20 years
(the first being described in 1969~\cite{McKay69}).
Blank describes accelerators for logic simulation, design rule checking,
placement, and routing. These are all tasks that remain computationally
expensive today, especially as digitial systems designs have grown in size
and complexity.

These accelerators were all dedicated hardware engines, capable only
of executing the specific application for which they were originally designed
and built. As a result, none of them were economically viable over any
length of time.  During this time period, the march of technology (specifically
the combination of Moore's Law and Dennard Scaling) quickly
overcame any functionality-limited design.  It simply wasn't long
before general-purpose machines became fast enough to eliminate the
performance advantage that was available via hardware specialization.

While the example give above was computer-aided design, there were similar
efforts in other domains (e.g., LISP machines~\cite{lisp,alphalisp},
Java machines~\cite{java,Schoeberl08}),
all of which suffered a similar fate.

Compare: GPU vs.~FPGA~\cite{bnw+10,cls+08,cmhm10,cz09,jpbc10,sww+10,tb10},
PRNG~\cite{thl09,tb09}, SPICE~\cite{kd09}, image processing~\cite{amy09}.
Productivity measures~\cite{jpbc10}.
Early review~\cite{bdh+10}.

The first known (to this author) combination of a graphics engine and
reconfigurable logic used on a common problem was presented
in two papers
by Kelmelis et al.~\cite{khdo06,kdh+06} in 2006.  They used a pair of PCI cards
(an EM Photonics Celerity card with a Xilinx Virtex-II FPGA and
an NVIDIA GeForce 7800 GTX graphics card) to accelerate the execution of
the finite-difference time-domain (FDTD) simulation of electromagnetic
waves~\cite{khdo06} and nanoscale devices~\cite{kdh+06}.


Development environment~\cite{cft+10},
included FPGA hardware but no empirical FPGA results~\cite{dy08}.

Cluster~\cite{tl10}, cluster (w/GRAPE)~\cite{sbm+09}.

Early applications~\cite{bkdb10,khdo06,shsc08,tl10}.

Map-reduce~\cite{ytt+08} (both, but not simultaneously).

