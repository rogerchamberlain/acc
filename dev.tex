\section{Development Tools}
\label{sec:dev}

The general responsibilities associated with application development
tools can be quite broad.  First, they might assist in the expression of
the computation to be executed (either by supporting one or more languages
or introducing a potentially new language).  Second, they might assist the
execution of the application, often through resource management and/or
scheduling.  We will consider each of these in turn.

\subsection{Expression of the Application's Computation}

Essentially, here we are interested in tools that help the application
developer express the computation that is to be performed. Essentially, what
must the programmer say to enable the execution platform to do what is
desired.  Traditionally, application expression using even a single
accelerator has been somewhat difficult (e.g., requiring the use of
low-level languages like Verilog and/or VHDL for reconfigurable logic
design), and the addition of a second accelerator with potentially vastly
different properties does nothing to make the task any easier.

Probably the easiest place to make progress on this front is the judicious
development of libraries.  By encapsulating either a portion of the
computation or some support function in a library, the application
developer is relieved of the responsibility to implemement that functionality.

As a first example, Thoma et al.~\cite{tdmp15} describe a framework for
supporting accelerator to accelerator communications (specifically supporting
data transfers between GPUs and FPGAs).  In particular, PCIe transfers
are made directly, device-to-device, without data being copied into
the main memory of the multicore host. 


Libraries:
VSIPL++~\cite{mlk12}.
CNNLab (for NN ML)~\cite{zlwx16}.

Language compilers:
OpenCL~\cite{Ahmed11}.
Liquid Metal~\cite{abb+12}.
Data-flow~\cite{szb+12},
FCUDA~\cite{pgs+13}.
OpenACC~\cite{lkv16}.

Us~\cite{blc17,cft+10,ctg+07,ftb+06,wcc12,wcc13}.
Luk~\cite{ttpl10}.

Choose accelerator based on properties of input:
QUARK~\cite{hcy+14,hjl+15}.

Target multiple accelerators with common code:
dense linear algebra~\cite{daa+15},
Kicherer et al~\cite{knbk12}.



EngineCL~\cite{dng+19}.

Resource management~\cite{bdm+13}.

Scheduling~\cite{kl17,lp17}.

\subsection{Design Space Exploration}

Luk~\cite{ll12,ll11,slkk13}.

UCR~\cite{bbg13}.

Scheduling (using simulator)~\cite{blby11}.
