\section{Conclusions and Future Directions}
\label{sec:conclude}

\subsection{Conclusions}

Over the past 15 years, there has been a growing interest in the use of
multiple accelerators on individual applications, the utilization of
\emph{truly} architecturally diverse systems.  The reasons for this
are many, but the dominant consideration is that the benefits achievable
through acceleration are strongly dependent upon the properties of the
application as well as the properties of the implementation.

Since applications are far from monolithic, by exploiting a different
accelerator for different parts of an application, one can achieve
better results overall than if the accelerator choice is limited to just
a single option.

The vast majority of recently developed applications that
exploit multiple accelerators deploy different portions of the application
to distinct accelerators.  Portions of the application that work well
on reconfigurable logic are deployed on reconfigurable logic, and portions
that work well on graphics engines are deployed on graphics engines.
This is not surprising, as each computational platform has its own
unique strengths and weaknesses.

While performance is still the dominant figure of merit, there is
increasing interest in energy consumption and/or energy efficiency as
an important consideration in how one wishes to deploy an application
in production use.  Accelerators often provide strong advantages 
in this circumstance, furthering the interest in their use.

\subsection{Future Directions}

%AutoTVM.

As an application developer with over a decade of experience using
architecturally diverse systems, there are a pair of things this
author would like to see achieved in the future. Both items directly
relate to the development tools available to authors of applications.

First, the compiler technology, especially for reconfigurable logic,
remains woefully behind the capabilities of compilers for traditional
software systems.  Even things such as the degree of loop unrolling,
which happens regularly and invisibly in software compilers, must be
explicitly specified by the developer for high-level synthesis.

The reason that parameters such as loop unrolling degree are exposed
to application developers is that we don't yet understand the
implications of the various choices prior to deployment and empirical
measurement.  In the software world, the compiler has a reasonably
well understood model of the implications of code generation choices
that it makes.  This is simply not yet the case in the domain of
high-level synthesis for reconfigurable logic.

Second, debugging tools are simply in their infancy. Today, an
application developer must use separate tools that are designed
for the individual accelerator(s) in use.  At the very least, they
have different user interfaces and modes of operation.  More
seriously, the visibility that they provide into the operations
of the application (especially those internal to an accelerator)
are quite deficient.

While there is clearly need for additional work, the fundamental
forces that make truly architecturally diverse systems attractive
are only getting stronger.  Dennard Scaling has essentially halted,
and Moore's Law is slowing down.  Architectural diversity is one
of the more promising approaches to addressing the challenges
inherent in this modern reality.
