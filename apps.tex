\section{Applications}
\label{sec:apps}

The vast majority of the applications that have been deployed on
more than one accelerator have combined graphics engines and reconfigurable
logic as the two types of accelerator.
Table~\ref{tbl:apps} summarizes the literature for these applications,
indicating the application itself, the year of publication (including
the citation), and various notes about the specific implementation.
Particularly, for each application it is indicated whether (or not)
the application assigns distinct functions to the two different
accelerator types, and also whether (or not) the map-reduce programming
model is used as part of the implementation.

%FPGA and GPU:
%nano sim~\cite{khdo06},
%MC sim~\cite{shsc08,tttl10},
%n-body sim~\cite{tl10},
%crypto~\cite{dfg+13},
%vision~\cite{bkdb10,ghgb11,brf14},
%astronomy (and discussion of dwarfs)~\cite{ibs12},
%astronomy (embedded in instrument)~\cite{kgb+14},
%cardiac optical mapping~\cite{mjk12},
%medical imaging~\cite{szb+12,sll13},
%image processing~\cite{dbd+13}.
%toy MC sim~\cite{admb14},
%BLASTp~\cite{Papad14},
%combo HPC and embedded~\cite{rpm+15},
%image processing in embedded application~\cite{enr+18,nsg+16,zxl+18},
%data acquisition in science~\cite{cab+17,vac+16},
%GWAS~\cite{kws+16},
%GWIS~\cite{wkhe17},
%driver assistance~\cite{wlhk17}
%gene-gene interactions~\cite{wkhe18},

\begin{table}[htp]
\centering
\caption{Applications deployed on multiple accelerators.}
\label{tbl:apps}
\vspace{0.1in}
\begin{tabular}{p{0.15\linewidth} | c | c | p{0.6\linewidth}}
Application & Ref. & Year & Notes \\ \hline
EM sim & \cite{kdh+06} & 2006 & First known example$^*$ \\ \hline
\multirow{4}{\linewidth}{Monte Carlo simulation} & \cite{ytt+08} & 2008 & European options pricing$^\dag$  \\ \cline{2-4}
 & \cite{shsc08} & 2008 & Financial value-at-risk computation$^*$  \\ \cline{2-4}
 & \cite{tttl10} & 2010 & Asian options and generalized asset pricing \\ \cline{2-4}
 & \cite{admb14} & 2014 & Computation of $\pi$$^*$ \\ \hline
\multirow{3}{\linewidth}{N-body simulation} & \cite{ytt+08} & 2008 & Dynamic workload distribution$^\dag$ \\ \cline{2-4}
 & \cite{tl10} & 2010 & Introduced Axel cluster$^\dag$ \\ \cline{2-4}
 & \cite{sm16} & 2016 & AWS GPUS plus local FPGA \\ \hline
\multirow{2}{\linewidth}{crypto} & \cite{ytt+08} & 2008 & Brute force key search$^\dag$ \\ \cline{2-4}
 & \cite{dfg+13} & 2013 & Password search for PDF documents$^*$ \\ \hline
\multirow{3}{*}{vision} & \cite{bkdb10} & 2010 & Pedestrian detection$^*$ \\ \cline{2-4}
 & \cite{ghgb11} & 2011 & Stereo 3D video production$^*$ \\ \cline{2-4}
 & \cite{brf14} & 2014 & Hand tracking$^*$ \\ \hline
\multirow{4}{\linewidth}{medical imaging} & \cite{szb+12} & 2012 & Includes both static and dynamic assignment$^*$ \\ \cline{2-4}
 & \cite{mjk12} & 2012 & Cardiac mapping \\ \cline{2-4}
 & \cite{sll13} & 2013 & Transmural electrophysiological imaging$^*$ \\ \cline{2-4}
 & \cite{rpm+15} & 2015 & Cone beam computed tomography$^*$ \\ \hline
\multirow{5}{\linewidth}{image processing} & \cite{dbd+13} & 2013 & Pedestrian recognition$^*$ \\ \cline{2-4}
 & \cite{rpm+15} & 2015 & Face recognition and character recognition$^*$ \\ \cline{2-4}
 & \cite{nsg+16} & 2016 & Data acquisition for experimental physics$^*$ \\ \cline{2-4}
 & \cite{enr+18} & 2018 & Add dynamic function distribution to above$^*$\\ \cline{2-4}
 & \cite{zxl+18} & 2018 & Increase frame rate to 9000 frames/s$^*$ \\ \hline
astronomy & \cite{kgb+14} & 2014 & Radio antenna cross-correlation$^*$ \\ \hline
\multirow{4}{\linewidth}{comp bio} & \cite{Papad14} & 2014 & Protein-protein sequence alignment$^*$ \\ \cline{2-4}
 & \cite{kws+16} & 2016 & Genome-wide association studies$^*$ \\ \cline{2-4}
 & \cite{wkhe17} & 2017 & Genome-wide interation studies$^*$ \\ \cline{2-4}
 & \cite{wkhe18} & 2018 & Computation of gene-gene interations$^*$ \\ \hline
video & \cite{rpm+15} & 2015 & Video coding and motion tracking$^*$ \\ \hline
\multirow{2}{\linewidth}{data aquisition} & \cite{vac+16} & 2016 & Linear accelerator$^*$ \\ \cline{2-4}
 & \cite{cab+17} & 2017 & Trigger systems for Large Hadron Collidor$^*$ \\ \hline
\multirow{2}{\linewidth}{driving} & \cite{wlhk17} & 2017 & Driver assistance (lane detection) \\ \cline{2-4}
 & \cite{lzh+18} & 2018 & Autonomous vehicle computations$^*$ \\ \hline
\end{tabular}
$^*$Assigns distinct functions to the two accelerators. $^\dag$Utilizes map-reduce.
\end{table}

An example of partitioning the workload based on the distinct functions to be
performed is described by Danczul et al.~\cite{dfg+13}. The authors are
attempting a brute force attack on the password for a set of PDF encrypted
files.  This requires both MD5 hash computations as well as RC4 stream cypher 
computations.  They assign the MD5 hashes to the graphics engines, in part
due to its limited memory requirements, and assign the RC4 computations to the
reconfigurable logic (in which the memory requirements of the RC4 algorithm
can be accommodated efficiently).

In a large fraction of the implementations where the workload is
divided by function onto each accelerator, the computation is 
organized as a pipeline.  Data flows into one accelerator, the portion
of the work assigned to that accelerator is performed, the output
of that pipeline stage then flows into the other accelerator, where
the portion of the work assigned to that accelerator is performed.

Sb\^{\i}rlea et al.~\cite{szb+12} extend this to general dataflow graphs,
supporting a richer set of topolgies than just a simple pipeline.  The authors
also incorporate the ability to support a work stealing scheduler across
execution platforms.

A clear benefit of the map-reduce paradigm is that it is relatively
straighforward to express the computational parallelism available in
an algorithm using this approach.  Several applications took advantage
of this.  In addition, it can also support the dynamic assignment of
workload across the available computational resources, illustrated as
early as 2008 by Yeung et al.~\cite{ytt+08}.

For the set of Monte Carlo simulations listed in the table,
four of the five problems attempted
(\cite{tttl10} includes two) are some form of financial computation.
This hints at the strong interest in accelerated computation on the
part of the financial industry.

Any time multiple compute resources are used in an application, there is
the potential for data movement between those resources to be a performance
bottleneck.  Bittner et al.~\cite{brf14} demonstrate direct graphics engine
to reconfigurable logic DMA data transfers over PCIe, avoiding a copy
to/from the host memory.

GPU and Phi:
Jacobi relaxation~\cite{cv16}.

%Move data to/from accelerator using compression (GPU and Phi)~\cite{bkp15}.

FPGA present, but part of infrastructure, not application~\cite{abb+13}.

Choose accelerator based on properties of input:
%linear algebra~\cite{gchg16,sll+13},
sparse matrix multiplication~\cite{gsbh16}. \FIXME{Add to table?}
%vision~\cite{mfo+16}.

%Compare FPGA with GPU:
%scientific computations~\cite{wghp11},
%Sliding-window computations~\cite{cfbs15},
%de novo assembly (comp bio)~\cite{mjk+16},
%optical flow~\cite{bnw+10}, quasi-MC sim~\cite{tb10},
%microscopy~\cite{tzwz15}, erasure coding (OpenCL)~\cite{czs+16},
%compressed sensing and Cholesky decomposition~\cite{ypl12},
%peak performance and sustained performance~\cite{vn14},
%energy efficiency, performance, and productivity~\cite{cdde13}.

Database join (common language: OpenCL)~\cite{rl17}.

%Compare FPGA, Cell, and GPU:
%matched filter~\cite{bgt07}, SPICE~\cite{kd09}, Smith-Waterman~\cite{bal+12}.

%Compare Cell and GPU:
%wavelet transforms~\cite{bck+11}, CT reconstruction~\cite{skkh07},
%PDE simulation~\cite{rd10},
%bioinformatics, molecular dynamics, and database~\cite{pts+12}.

%Compare FPGA, GPU, and Phi:
%sparse matrix multiplication~\cite{gsbh16},
%predictive models for power and energy~\cite{opr+17}.

%Compare GPU and Phi:
%microscopy~\cite{tkk+14}, Ising model~\cite{ws13},
%quantum many-body simulation~\cite{Lyakh15},
%quantum chemistry~\cite{lrg14},
%productivity, performance, and energy~\cite{mlp+17}.
