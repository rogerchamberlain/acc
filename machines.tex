\section{Machines}
\label{sec:machines}

After the initial set of machines described in the previous section,
a number of additional machines have been designed, built, and deployed
that support multiple accelerators.

In 2012, the Chimera system was described~\cite{ibs12}.
It combined Intel multicores, NVIDIA Tesla C2070 graphics engines,
and an Altera (now Intel) Stratix-IV reconfigurable logic card using
the motherboard's PCIe bus. The authors describe approaches to deploying
a variety of applications on the system. However, performance results
are not provided.

A multiple accelerator cluster built at the Center for Development of
Advanced Computing (C-DAC) in Bangalore, India, was described in
2014~\cite{admb14}. Here, the authors adapted the StarPU development
environment~\cite{starpu} to support reconfigurable logic as well as
simultaneous support of CUDA and OpenCL in the same application.
A Monte Carlo simulation application is described, achieving a 22$\times$
speedup relative to a Xeon processor.

Also in 2014, Wu et al.~\cite{whk+14} describe a heterogeneous compute
platform that has multicores, graphics engines, and reconfigurable logic
that has a focus on power efficient computing.
The authors explore four application examples, but similar to the QP
machine, chose to exploit one accelerator type or another, but not both
simultaneously.

Cloud~\cite{pcc14}.
Embedded~\cite{rpm+15}.
